\documentclass[10pt,pdf,aspectratio=169]{beamer}
\usepackage[T2A]{fontenc}       %поддержка кириллицы
\usepackage[utf8]{inputenc}
\graphicspath{{./pictures/}{/home/eddy/}}
\usetheme{Boadilla}
\usefonttheme{structurebold}
\usefonttheme[onlymath]{serif}
\setbeamercovered{transparent}

\setbeamercolor{color1}{bg=blue!90!black,fg=white}
\setbeamercolor{normal text}{bg=black,fg=cyan}
\setbeamercolor{frametitle}{fg=red,bg=cyan}

\title[короткое название]{Длинное название презентации}
\subtitle[кор. подназв.]{Длинное подназвание}
\author[А.В. Тор]{Арсений Владимирович Тор}
\institute[НИИХЗ]{НИИ Химии Земли\\
	{\tiny Лаборатория радионуклидного анализа}\\
}
\date[год]{Место проведения конференции, год.}

\begin{document}
\begin{frame}[plain]
\maketitle
\end{frame}
\begin{frame}\frametitle{План}
\tableofcontents
\end{frame}

\section{Секция 1}
\subsection{Подсекция 1}
\begin{frame}
	\transwipe
	\frametitle{Влияние чего-то на что-то}
	\begin{columns}
		\column{0.3\textwidth}
		\begin{block}{Примеры видимости}
			\onslide<1>{Непрозрачный на 1}\par
			\only<2>{Появится лишь на 2}\par
			\visible<3>{Лишь на 3}\par
			\invisible<4>{на 4 пусто}\par
			\alt<2>{слайд 2}{что-то явно не слайд 2}\par
			\temporal<2>{первый}{второй}{третий (с копейками)}\par
			\uncover<3>{нет меня, я --- 3}\par
		\end{block}
		\column{0.6\textwidth}
		\only<1,4>{
			\begin{block}{Блогозаголовок}
				А вот вам!
			\end{block}
		}
		\textbf{Нумерованный список с оверлеями}:
		\begin{itemize}
			\item<1-> Этот пункт будет показан всегда
			\item<2-> Этот -- начиная со второго кадра
			\item<1,3> Этот -- на первом и третьем кадрах
		\end{itemize}
	\end{columns}
\end{frame}

\subsection{Подсекция 2}
\begin{frame}
	\frametitle{Влияние кого-то на что-то}
	\begin{block}{формула}
		Содержимое блока, дурацкая формула:
		$$\int_1^2\sin x\,dx$$
		А дальше - будем показывать формулу по кускам. Эдакая порнография эпохи "зухелей"...
		$$
		\sin^2x\pause+\cos^2x\pause=1
		$$
	\end{block}
\end{frame}

\section{Секция 2}
\subsection{подсекция}
\begin{frame}
	\frametitle{Когда уже эта презентация кончится?}
	\only<1-3>{
		\begin{beamercolorbox}[shadow=true, rounded=true]{color1}
			С первого по третий --- рассчитайсь!
		\end{beamercolorbox}
	}
	\only<2->{
		\begin{block}{}
			Со второго кадра я здесь. Но не жадничаю. Место изначально не занимаю.
		\end{block}
	}
	\begin{columns}
		\column{0.5\textwidth}
		\onslide<3>{
			\begin{alertblock}{Ох уж эти объявления!}
				Не делайте такие презентации! Больше 80 слов на слайд | уйма свистоперделок |
вырвиглазные цвета | множество блоков == смерть!
			\end{alertblock}
		}
		\column{0.5\textwidth}
		\onslide<3>{
			\begin{exampleblock}{Пример}
				А меня долго ждали. Но не показывали.
			\end{exampleblock}
		}
	\end{columns}
\end{frame}
\newcount\ooo
\newdimen\offset
\begin{frame}
	\animate<2-20>
	\animatevalue<1-20>{\ooo}{100}{0}
	\animatevalue<1-20>{\offset}{0cm}{0.7\textheight}
	\vbox to \textheight{
	\frametitle{Вот и сказочке конец, а кто слушал --- молодец!}
	\begin{colormixin}{\the\ooo!averagebackgroundcolor}
	\vskip\offset\centering{Спасибо за внимание!}
	\end{colormixin}}
	\transduration{0.05}
\end{frame}
\end{document}
